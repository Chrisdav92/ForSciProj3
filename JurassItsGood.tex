\documentclass[12ptletterpaper]{paper}
%this is where the images come from
\usepackage{graphicx}
\graphicspath{ {images/} }
%tab command
\newcommand\tab[1][1cm]{\hspace*{#1}}
\setlength{\oddsidemargin}{-0.25in} % Left margin of 1 in + 0 in = 1 in
\setlength{\textwidth}{7in}   % Right margin of 8.5 in - 1 in - 6.5 in = 1 in
\setlength{\topmargin}{-.75in}  % Top margin of 2 in -0.75 in = 1 in
\setlength{\textheight}{9.2in}  % Lower margin of 11 in - 9 in - 1 in = 1 in
\linespread{1.5}
\usepackage{indentfirst}
\usepackage[authordate]{biblatex-chicago}
\usepackage[bottom]{footmisc}
\usepackage{hyperref}
%this is for quotations
\usepackage{epigraph}
\hypersetup{
	colorlinks,
	citecolor=black,
	filecolor=black,
	linkcolor=black,
	urlcolor=black
}
% --- create title format ---
\title{
	\begin{center}
		\normalfont \normalsize
		\rule{\linewidth}{.5pt} \\[0.4cm] 
		\huge {Jurassic Park} \\ 
		\small{Chris P David}\\
		{\today}\\
		{CSC 317}\\
		{Professor MacDonald}
		\rule{\linewidth}{.5pt} \\
	\end{center}
}
% --- end title format ---


\begin{document}
	\begin{titlepage}
		\clearpage
		\maketitle
		\thispagestyle{empty}
	\end{titlepage}
	\pagebreak	
	\tableofcontents
	\begin{flushleft}
		\pagebreak
		\section{Introduction}
		\tab For this paper I will be discussing the merits in terms of technology of the film Jurassic Park alongside any views about technology during that time period and further discussing any technological inaccuracies as well . I'm choosing this film as this was the first movie I ever saw when I came to America, as well as being a huge nerd, I loved dinosaurs growing up and now that I'm in a tech field I would love to see what really was accurate and what was Hollywood magic. I will also be referencing the book as well as I read it along side the other movies, this is purely for any supplemental information.
		
		\section{Technology during this time}
		\epigraph{"Dr. Grant, my dear Dr. Sattler... Welcome to Jurassic Park"}{John Hammon}
		\tab When Jurassic Park debuted In 1993, the world was experiencing a massive tech growth. Three years pior we just launched the "World Wide Web", technically really HTML, which was a major step in the way we now communicate and share data with another. It brought the first host systems,servers and the client application that we use, the web browser\footnote[1]{\hyperlink{Technological Advances of the 90s}{Technological Advances of the 90s}}. We also saw the rise of Microsoft on Jan 10 1990\footnote[2]{\hyperlink{Timelines}{Timelines}} as well as the start of mobiles phones being present in the world. We were really starting to go towards the path that we basically are on at this moment of time. We also had the release of the Pentium processor (1993) which was the workhorse of Intel chip for many years to come \footnote[3]{\hyperlink{Pentium}{Pentium}}.We started to break the old world tech of computers only talking to themselves, to the processing power computers could do in the 80's compared to the 90's as well as memory sizes increasing, from the 10MB Rodime RO352 to the Mustang 1820 20 megabyte 1.8" hard drive\footnote[4]{\hyperlink{Processing-power}{Processing-power}}  \footnote[5]{\hyperlink{Prep}{Prep}}. \\
		\tab Technology was also making breakthroughs in the science community as well with the Genome project (which heavily impacted this book). In terms of the book it was released in 1990 by Michael Crichton, and surprisingly was up to date in terms of what was present in the tech world and not too abstract. 
		
		\section{Events that have factual backing}
		\epigraph{"I am not a computer nerd. I prefer to be called a hacker!"}{Lex Murphy}
		
		
		\tab During this section I will be discussing two events that happened in the movie that actually have factual information associated with them. One fact that I want to say here was that Steven Spielberg wanted the raptor's to be 10 feet tall, bigger than they were previously found to be. After the movie, archaeologist's found the Utah raptor which at the time of being discovered was around 8 feet in height and around, you guessed it, 10 feet in length\footnote[6]{\hyperlink{Raptor}{Raptor}}. So when watching/reading the movie/book I figured the most realistic examples I could pull would be the way they depicted the dangers of over reliance of technology, why there should be safeguards, and most interestingly the depiction of the dinosaurs (though there are some inaccuracies).
		
		\subsection{Over-reliance of Technology}
		\tab One of the main themes of this book was the over reliance of technology. The main cause of the book/movie was the failure of the security system caused by the system being shut down by the character Dennis Nedry. The reason for this event being a realistic one, is that altering a program or shutting off a software could potentially cause massive side effect which could result in injuries or in worst cases death. Some events in which this did happen in real life where the Russian Gas pipe explosion in 1982, and the Patriot system of 1991
		
		
		\tab This event, which was the cause of a Trojan Horse added by the CIA (allegedly), was when the Trojan was only supposed to cause the pipeline to shutdown but a error in the programming instead caused an explosion to occur. It is to be noted that this is the largest man-made non-nuclear explosion in Earth's history\footnote[7]{\hyperlink{Russian Gas Explosion}{Russian Gas Explosion}}.This explosion caused millions to the economy (Russia), but was underplayed as Russia said, in a official statement, that it was an issue on their part.
		
		\tab The Patriot Missile failure is a famous example of software going wrong in that essentially the system incorrectly failed to track a missile aimed towards the base. The result was, in total 100 soldiers being hurt with another 28 being killed. In the report done by the General Accounting office, the issue was due to the calculation of time by the program.The issue being the calculation of  the time starting from boot, the time was being calculate using a 24-bit fixed point register, but due to the calculation, was being chopped after the 24bit. This "small" error in conjunction with being multiplied by large numbers, lead to a very inaccurate representation of time. This lead the program to think a missile traveling at 1.676 M/S was outside the "range" that was safety allowed\footnote[8]{\hyperlink{Patriot Missile}{Patriot Missile}}.\tab To sum up this section it's safe to say that not all software or systems will work 100 percent of the time. It's in fact better to say that a system will work as intended as long as the parameters are set and that calculations are correct as well. The events I described where results of mistakes that could of been avoided if those that worked on them did numerous test on them.
		\subsection{Depiction of Dinosaurs}
		\tab Dinosaurs were portrayed to the best of their image for that time. As I said above with Steven Spielberg's interpretation of the raptor's, the dinosaurs where done relatively great. One of the production advisor was the famous archaeologist Jack Horner, he helped with some of the thought process of how the dinosaurs worked, and how they may of looked. One of the most notable points of the flim is when Grant,Sam and Lex are in the open field and see a "herd" of Gallimimus. Grant goes to say how they are moving as a pack of birds. Since the movie came out it's been now firmly established that many birds of prey and like are direct descendants of dinosaurs. A great example that I will use is one of larger birds,the Cassowary. It has been described as a modern day raptor. The characteristics of the Cassowary are that they are a very large bird often getting up to  137 pounds and up to 6.5 feet tall (that is very scary when you realized they are the third smallest of the big birds,only in height). 
		\tab Besides the bird-like tendencies, the film does a great job of also showing what the dinosaurs looked like. During this time period Dinosaurs where thought to have dull colors like they have in the film. Not only that but the depiction that they are warm-blooded was backed by research, to which Jack Horner was quoted saying "The vasculature (number of blood vessels, revealed by microscopy of bone structure) and growth rates of baby dinosaurs required warm innards....There's nothing alive today that has the vasculature of a dinosaur that is cold-blooded."\footnote[8]{\hyperlink{Jurassic Science}{Jurassic Science}}
		\tab After all that has been said Jurassic Park did a very good job depicting dinosaurs to the best of the archaeologists that helped with the flim, even if there where some exaggeration of the way dinosaurs acted. (It still is very unknown how dinosaurs truly lived, though we have some idea).
		
		\section{Inaccuracies regarding Jurassic Park}
		\tab We go on to this section in which I will explain three inaccuracies that happened during the film, and I will keep them more on the tech side of things.
		 What I did was to watch the movie again and sort of jot down some things that to me where incorrect or where I think they did some Hollywood magic to appeal to the crowd.
		 
		\subsection{Unix OS}
		\epigraph{"It's a UNIX system! I know this!"}{Lex Murphy}
		\tab One of the big issues that is seen in this film is the way one the main character's, Lex, was able to get into the system by some on-screen hacking. Although it is possible to get into the root system on Unix. It would still be very hard to obtain full control of a professional system, especially one that has very up to date securities, but based on the way the system was structured I could see this happening.  Another issue that I thought was real was when Lex goes though the system . We see her click and "travel" though the park. Although the file structure would be structured like this, I thought it would be almost impractical and also hard in my opinion to have a UI designed to looked like the way the film portrays it. But after further research into the GUI I found out that it was a a real Unix system, which was called a Silicon Graphics workstation. What was used in the movie was a real demo application.So really she wasn't really doing anything just showing a demo\footnote[9]{\hyperlink{Movie Science}{Movie Science}} There are still inaccuracies with this depiction still but it isn't to say that at least they did an ok job in that they showed "hackers" not just being older male person but a teenager who happened to be a girl!
		\subsection{Workplace ethics}
		\tab The second issue that takes place in this film is the fact that once the system was compromised, they weren't able to go around the program that Nedry placed into the system. One of the things that immediately popped into my mind when watching the film was why wasn't there a default Administration account that would allow Arnold to bypass any program being run and force stop it, as well as if it was a program being run why not be able to force shutoff the computer as well? One of the main issues as well is the idea of a employee having too much power or abilities. In this case it was Nedry, who was able to control the whole park form his terminal. In this case especially there should not be one employee who has complete access to an entire mainframe, it should be multiple persons with varying level security access. But also it should be in place that even if you given a person access to many different areas, there should still be a sort watchdog program or a person who's task it is to make sure company computer's or systems aren't being abused.
		
		\subsection{Contingency plans}
		\epigraph{"Clever Girl"}{Muldoon}
		\tab Lastly the underlying issue that occurs in this flim is the lack of a backup/power option. In the film after the software causes the cages to go off and systems to fail, Samuel Jackson's character Ray Arnold has to go to the other side of the facility to restore the power (He fails to reset the switches and was killed). The glaring issue is that usually if you have a huge data center or any similar sort of area you would have multiple power options in case the main fails. This would avoid any data/time to be wasted by having the reset option by readily available and/or by having a autonomous switch to axillary power.
		\section{Conclusion}
		\tab In conclusion I still love Jurassic Park to this day. With it's excellent use of puppeteering of the dinosaurs (which should definitely have a comeback), amazing ideas about technology and how it should be used, to the overall themes of Man playing God and the chaos theory, it remains on my list of top movies you should watch. Though it has some insane ideas of genetically modified organisms at that time, we have gotten to a point where a majority of the ideas presented in the film have come to fruition. Shortly after the movie was released we successfully cloned a sheep, as well as many other animals. We have unlocked the human gene and even brought a animal back form extinction (the guar)\footnote[10]{\hyperlink{Jurassic Science}{Jurassic Science}}. Yes there are some gaping plot-holes in the movie but in essence thats what it is after all, a fantastic book turned movie. I'll end this paper with one of my favorite jokes/quotes and it's as follows:\epigraph{"Tim: What do you call a blind dinosaur?
			
			Dr. Alan Grant: I don't know. What do you call a blind dinosaur?
			
			Tim: A Do-you-think-he-saurus.
			
			Dr. Alan Grant: Ha ha. Good one.
			
			Tim: What do you call a blind dinosaur's dog?
			
			Dr. Alan Grant: You got me.
			
			Tim: A Do-you-think-he-saurus Rex."}{Tim, Dr.Alan Grant}
		\pagebreak
		\section{Bibliography}
		\noindent\hypertarget{Technological Advances of the 90s}{Technological Advances of the 90s.\\. (2012). 1990s Technology Timeline: Massive Growth During the Decade of the 90s. Retrieved September 26, 2016, from http://www.brighthub.com/education/homework-tips/articles/123405.aspx} \vspace{12pt}
		
		\noindent\hypertarget{Timelines}{Timelines.\\Technology Advancements of the 90's timeline. (n.d.). Retrieved September 26, 2016, from https://www.timetoast.com/timelines/technology-advancements-of-the-90s} \vspace{12pt}
		
		\noindent\hypertarget{Pentium}{Pentium\\Kiger, B. P. (2014). The '90s: Science and Technology. Retrieved September 28, 2016, from http://channel.nationalgeographic.com/the-90s-the-last-great-decade/articles/the-90s-science-and-technology/} \vspace{12pt}
		
		\noindent\hypertarget{Processing Power}{Processing Power\\FLOPS (n.d.). Processing Power Compared. Retrieved September 27, 2016, from http://pages.experts-exchange.com/processing-power-compared/} \vspace{12pt}
		
		\noindent\hypertarget{Prep}{Prep\\Rodime RO352 10MB 3.5IN HH MFM Free Econony Shipping in US. (n.d.). Retrieved September 28, 2016, from http://www.4drives.com/4drives/RO352.htm} \vspace{12pt}
		
		\noindent\hypertarget{Raptor}{Raptor\\Trivia. (n.d.). Retrieved September 27, 2016, from http://www.imdb.com/title/tt0107290/trivia} \vspace{12pt}
		
		\noindent\hypertarget{Russian Gas explosion}{Russian Gas explosion\\DevTopics. (n.d.). Retrieved September 28, 2016, from http://www.devtopics.com/20-famous-software-disasterss} \vspace{12pt}
		
		\noindent\hypertarget{Patriot Missle}{Patriot Missle\\The Patriot Missile Failure. (n.d.). Retrieved September 27, 2016, from https://www.ima.umn.edu/~arnold/disasters/patriot.html} \vspace{12pt}
		
		\noindent\hypertarget{Jurasic Science}{Jurasic Science\\Jurassic Park Science. (n.d.). Retrieved September 27, 2016, fromhttp://www.livescience.com/37297-science-of-jurassic-park-evolved.html} \vspace{12pt}
		
		\noindent\hypertarget{Movie Science}{Movie Science\\Science of Movies. (n.d.). Is the Unix operating system featured in Jurassic Park real? Retrieved September 29, 2016, from http://movies.stackexchange.com/questions/9745/is-the-unix-operating-system-featured-in-jurassic-park-real} \vspace{12pt}		
		
		
	\end{flushleft}
\end{document}